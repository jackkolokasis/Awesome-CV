Big data applications critically differ from traditional ones in that
the amount of temporary and persistent information is often more than
an order of magnitude larger than the input dataset. At the same time,
DRAM scaling is ending, and current data movement trends dictate
data placement close to the processor. My research explores
heterogeneous memory and storage hierarchies for scalable managed
big data analytics. My particular focus is on integrating emerging
non-volatile memory (NVM) and fast storage devices such as
NVMe SSDs in the analytics stacks.

To fully exploit the potential of emerging memory and storage devices
requires cross-layer memory management that involves user
applications, managed language runtimes, and the operating system. I
have been working on one such system as part of my Ph.D. research that
enables the transparent use of multi-tier terabyte-scale heaps for
Apache Spark over DRAM and NVMe storage. I proposed TeraCache that
provides large compute caches for storing intermediate resilient
distributed dataset (RDD) objects in Spark. My work resolves the
tension between garbage collection (GC) and
serialization/deserialization by providing a memory-mapped heap (MMH)
over storage devices. It eliminates both GC and serialization
overheads without impacting the Java memory model.

Enabling MMH over fast storage brings many challenges which my work
tackles. (1) TeraCache provides a large virtual address space for Java
applications by mapping a second heap over a storage device. The
second heap is not garbage collected. My work exploits Spark semantics
to identify long-lived RDDs.  (2) TeraCache collects the two heaps in
isolation including tracking and managing references across the memory
and storage heaps.  (3) TeraCache removes serialization by providing
direct access to cached RDDs in the second heap. 

I strongly believe my research can potentially improve the efficiency
of managed language runtimes for datacenters workloads. My research
focuses on the challenges faced by memory-intensive services at
Twitter and other enterprises that regularly deal with big data.  I
would welcome the opportunity to examine my proposed techniques on
real-world applications and infrastructure and interact with experts
at Twitter.
